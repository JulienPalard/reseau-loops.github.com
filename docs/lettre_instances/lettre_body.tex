En mai dernier, le réseau \href{http://reseau-loops.github.com/}{LoOPS}
a été créé. LoOPS est le réseau local
impliquant potentiellement tous les personnels ayant une activité de
développemennt logiciel dans les établissements d'enseignement supérieur
et de recherche de la zone "Paris Sud-Ouest", qu'il s'agisse de personnels
techniques, d'enseignants ou de chercheurs.

La mission fondamentale du réseau est de faciliter le partage de pratiques,
savoir-faire et connaissances entre développeurs dans le but de permettre à
chacun de progresser dans son activité de développement logiciel.

D'autres \href{http://devlog.cnrs.fr/region}{réseaux locaux similaires} existent dans d'autres régions
et sont affiliés au réseau métier national des développeurs de logiciels
\href{http://devlog.cnrs.fr}{Devlog} reconnu par la Mission Ressources et Compétences Technologiques
(\href{http://www.mrct.cnrs.fr/}{MRCT}) du CNRS, par l'INRIA et l'INRA.

A l'heure actuelle, une
\href{http://reseau-loops.github.com/journee\_2012\_05\_31.html}{première journée}
de rencontres a été organisée par
LoOPS le 31 mai dernier et a réuni une cinquantaine de développeurs de
différents établissements (CNRS, INRIA, Université Paris-Sud, CEA,
Polytechnique, Soleil,~\dots).
Une \href{http://reseau-loops.github.com/journee\_2012\_12.html}{seconde journée} aura lieu le 11 décembre
prochain. Ces journées sont l'occasion d'échanges, de veille et de
formation pour les participants.

D'autres événements sont également organisés ponctuellement dans chaque
établissement (séminaires, exposés, ateliers, ...). Ces actions participent
de la formation des développeurs de l'ESR de la zone "Paris Sud-Ouest". LoOPS
a donc également vocation à favoriser la circulation d'informations concernant
ces actions.

Pour que le réseau puisse assurer ses missions reconnues utiles par la MRCT,
nous souhaiterions:
\begin{itemize}
\item que la participation au réseau de vos personnels impliqués dans le
  développement logiciel soit encouragée,
\item pouvoir bénéficier de soutiens financier, humain et logistique pour
  l'organisation de prochaines journées LoOPS (restauration, salles,
  inscriptions, annonces, ...).
\end{itemize}

Nous effectuons la même démarche auprès des autres établissements concernés
(INRA, INRIA, CNRS, Universités, CEA, ...).

Bien évidemment, nous sommes disponibles pour vous rencontrer afin de vous
exposer plus en détail notre démarche et nos besoins.